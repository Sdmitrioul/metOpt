\begin{large}
	\textbf{Функция 2 - $f(x,y) =  x^2 + 3 y^2-20 x-12 y+5$:}
\end{large}
\bigskip

\textbf{Метод градиентного спуска}

Количество итераций: 20.
		
\begin{figure}[h]
	\center{\includegraphics[width=0.4\linewidth]{static/fun-2-g.png}}
	\caption{Hешение 2-ой  функции градиентным спуском}
\end{figure}
Градиентный спуск даже при выборе достаточно благоприятной функции, имеет большое количество итераций за счет постоянного шага. Это отчетливо видно на визуализации.
\bigskip
\textbf{Метод наискорейшего спуска}

Количество итераций: 19.

\begin{figure}[h]
	\center{\includegraphics[width=0.4\linewidth]{static/fun-2-f.png}}
	\caption{Решение 2-ой  функции методом наискорейшего спуска}
\end{figure}
Метод наискорейшего спуска на второй функции показал же не такое большое преимущество перед градиентным спуском, количество итераций при приближении к минимуму даже на равномерной функции достаточно сильно увеличивается. вследствии чего выигрыш начального быстрого приближения становится малозначимым .