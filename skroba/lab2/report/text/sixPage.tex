\textbf{Метод наискорейшего спуска}
	
\begin{figure}[h]
	\center{\includegraphics[width=0.5\linewidth]{static/fast.png}}
	\caption{График зависимости наискорейшего спуска}
\end{figure}
Метод наискорейшего спуска так же показывает, что разнится в количпестве итераций от размерности незначительна и близка к константному значению. При этом при достаточно большом числе обусловленности так же идет резкий рост количества итераций, что связано с погрешностью возникающих при вычислениях. В свою очередь, метод показывает меньшее количество итераций, но "реальное" время его работы особенно при больших размерностях пространства, сильно превышает время работы метода градиентного спуска, что связано с большим количеством вычислений при решении задачи поиска оптимального шага на направлении.

\bigskip

\textbf{Метод сопряженных градиентов}
	
\begin{figure}[h]
	\center{\includegraphics[width=0.5\linewidth]{static/conjugate.png}}
	\caption{График зависимости сопряженных градиентов}
\end{figure}
Метод сопряженных градиентов, как и прежде, показывает наименьшее количество итераций среди всех приведенных методов. Но в отличии от предыдущих рассмотренных методов количество итераций зависит от размерности пространства.

\bigskip
