\begin{center}
	\begin{Large}
		\textbf{4. Выводы и сравнения}
	\end{Large}
\end{center}

\bigskip

Методы градиентного спуска и наискорейшего спуска показывают достаточно показывают, что для поиска минимума им обоим требуются достаточно большое количество итераций. При этом Метод градиентного спуска равномерно ищет минимум на всех этапах(ускоряясь при уменьшении шага), а метод наискорейшего спуска показывает себе в начале поиска, что дает возможность для использования его чтобы быстро вычислить окрестность минимума функции, а после найти его приближении при помощи другого метода. Оба показывают слабое влиения на время вычисления размерность функции. Но на реальных испытаниях, хоть и метод наискорейшего спуска имеет на 50 процентов  меньше итераций, реальное время его работы сильно больше при большой размерности пространства. Метод сопряженных градиентов показал себя наиболее хрошо из всех методов, и количество итераций и реальное время работы было меньше, чем у предыдущих, н оза счет зависимости от размерности пространства вероятно, на достаточно большом по размерности пространстве он начет уступать другим методам. При этом он намного лучше ведет себя при увеличении числа обусловленности функции по сравнению с другими методами.

\bigskip

\textbf{Вывод:} За счет использования ортоганальных направлений метод сопряженных градиентов показывает наилучшее время работы на не слишком больших размерностях пространств. В свою очередь комбинирование методов наискорейшего спуска и метод градиентов может дать достаточно хорошее время сходимости функции.

\begin{center}
	\begin{Large}
		\textbf{5. Приложения}
	\end{Large}
\end{center}

Реализация всех методов и всех вспомогательных классов выводящих резальтаты представленна в репозитории (\href{https://github.com/Sdmitrioul/metOpt/tree/main/skroba}{sdmitrioul})
