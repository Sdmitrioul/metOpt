\textbf{Метод сопряженных градиентов}

Количество итераций: 3.
	
\begin{figure}[h]
	\center{\includegraphics[width=0.5\linewidth]{static/fun-2-c.png}}
	\caption{Решение 2-ой  функции методом сопряженных градиентов}
\end{figure}
Метод сопряженных градиентов на второй функции тоже показал хороший результат, найдя минимум всего за 2 итерации.
\bigskip

\begin{large}
	\textbf{Зависимость количества итераций от размерности пространства и числа обусловленности функции}
\end{large}

\bigskip
Для изучения зависимости был написан генератор диагональных матриц с заданным числом обусловленности и размерности.
\bigskip

\textbf{Метод градиентов спуска}
	
\begin{figure}[h]
	\center{\includegraphics[width=0.5\linewidth]{static/gradient.png}}
	\caption{График зависимости градиентов спуска}
\end{figure}
Метод градиентного спуска показывает, что разнится в количпестве итераций от размерности незначительна и близка к константному значению. При этом при достаточно большом числе обусловленности идет резкий рост количества итераций, что связано с погрешностью возникающих при вычислениях. 
\bigskip
