\textbf{Метод наискорейшего спуска}

Количество итераций: 56.

\begin{figure}[h]
	\center{\includegraphics[width=0.4\linewidth]{static/fun-1-f.png}}
	\caption{Решение 1-ой  функции методом наискорейшего спуска}
\end{figure}
Метод наискорейшего спуска позволяет ускорить процесс поиска минимума, за счет поиска оптимального шага на каждой итерации. Это отчетливо видно на визуализации поиска, на нем отсутствуют частые и длинные шаги, при этом скорость поиск остается относительно равномерной все время работы алгоритма, что бозволяет уменьшить количество итераций в половину.
\bigskip
\textbf{Метод сопряженных градиентов}
	
Количество итераций: 3.
		
\begin{figure}[h]
	\center{\includegraphics[width=0.4\linewidth]{static/fun-1-c.png}}
	\caption{Решение 1-ой  функции методом сопряженных градиентов}
\end{figure}
Метод сопряженных градиентов вследствии правильного выбора направлений и шага находит минимум функции достаточно быстро(за конечное время от размерности пространства), при этом на него почти не влияют особенности функции.

