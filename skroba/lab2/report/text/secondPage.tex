\begin{large}
	\textbf{Функция 1 - $f(x,y) = 16x^2+ 8y^2 - 20xy + 5x - 7y + 1$:}
\end{large}
\bigbreak
\textbf{Метод градиентного спуска}

Количество итераций: 119.
	
\begin{figure}[h]
	\center{\includegraphics[width=0.6\linewidth]{static/fun-1-g.png}}
	\caption{Hешение 1-ой  функции градиентным спуском}
\end{figure}

Визуализация показывает недостатки метода градиентного спуска, в частности из-за достаточного большого начального шага, поиск происходит достаточно рванно, те пока не произойдет уменьшение шага последующие итерации будут иметь малое влияние на уменьшение функции. Скорость уменьшения функции будет снижаться после каждой итерации до уменьшения шага, это и дает достаточно большое количество итераций для поиска.
