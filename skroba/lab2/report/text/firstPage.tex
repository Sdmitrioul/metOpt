\begin{center}
	\begin{Large}
		\textbf{1. Постановка задачи}
	\end{Large}
\end{center}

\begin{itemize}
	\item Реализовать алгоритмы многомерной оптимизации функций:
			\begin{enumerate}
				\item Метод градиентного спуска
				\item Метод наискорейшего спуска
				\item Метод сопряженных градиентов
			\end{enumerate}
	\item Оценить различия скорости сходимости методов в зависимости от: метода оптимизации, входных данных, в частности от числа обусловленности оптимизируемой фунции и размерности пространства.
	\item Построить графики зависимости и визуализации работы методов. 
\end{itemize}

\begin{center}
	\begin{Large}
		\textbf{2. Исследуемые функции}
	\end{Large}
\end{center}


\begin{itemize}
	\item Функция 1: $f(x,y) = 16x^2+ 8y^2 - 20xy + 5x - 7y + 1$
	\item Функция 2: $f(x,y) =  x^2 + 3 y^2-20 x-12 y+5$
\end{itemize}

\begin{large}
	\textbf{Аналитические решения функций:}
\end{large}

\begin{itemize}
	\item \textbf{Функция 1:}
	$f(x,y) = 16x^2+ 8y^2 - 20xy + 5x - 7y + 1$\\
	\begin{equation*}
 		\begin{cases}
   			f_{x}'=0 \\
   			f_{y}'=0
 		\end{cases}
 		\Leftrightarrow minimum f(x, y)
	\end{equation*} \\
	\begin{equation*}
 		\begin{cases}
   			f_{x}'=32x - 20y + 5 = 0 \\
   			f_{y}'=16y - 20x - 7 = 0
 		\end{cases}
 		\Leftrightarrow
 		\begin{cases}
   			x = - \frac{15}{28} \\
   			y = \frac{31}{28}
 		\end{cases}
 		\Leftrightarrow
 		\min f(x,y) = - \frac{43}{28}
	\end{equation*}
	
	\begin{figure}[h]
		\center{\includegraphics[width=0.6\linewidth]{static/fun-1-wolfram.png}}
		\caption{Аналитическое решение 1-ой  функции и ее график}
	\end{figure}
	
	\item \textbf{Функция 2:}
	$f(x,y) = x^2+3y^2-20x-12y+5$\\
	\begin{equation*}
 		\begin{cases}
   			f_{x}'=0 \\
   			f_{y}'=0
 		\end{cases}
 		\Leftrightarrow minimum f(x, y)
	\end{equation*} \\
	\begin{equation*}
 		\begin{cases}
   			f_{x}'=2x - 20 = 0 \\
   			f_{y}'=6y - 12 = 0
 		\end{cases}
 		\Leftrightarrow
 		\begin{cases}
   			x = 10 \\
   			y = 2
 		\end{cases}
 		\Leftrightarrow
 		\min f(x,y) = -107
	\end{equation*}
	
	\begin{figure}[h]
		\center{\includegraphics[width=0.6\linewidth]{static/fun-2-wolfram.png}}
		\caption{Аналитическое решение 2-ой  функции и ее график}
	\end{figure}
\end{itemize}

\bigskip

\begin{center}
	\begin{Large}
		\textbf{3. Полученные результаты}
	\end{Large}
\end{center}

\textbf{Параметры исследований:}
\begin{itemize}
	\item[-] Условие остановки: $||\nabla f (x^{[k]})|| \leq \epsilon$; 
	\item[-] Начальная точка: $x^0 = (0, 0, ..., 0)$;
	\item[-]  $\epsilon = 10^{-5}$;
	\item[-]  $\delta = 0.95$;
	\item[-] Начальный шаг для градиентного спуска: 1;
	\item[-]  Число итераций для обновления метода сопряженных градиентов: $10$.
\end{itemize}
