\textbf{-Сравнение работы методов Гаусса и метода основаного на LU-разложении}
\\
\bigskip
\begin{tabular}{|c|c|c|c|c|c|c|c|}
\hline
n & k & LU & $ || x^{*} - x_{k} || $ & $ || x^{*} - x_{k} || / || x^{*} || $ & Gauss & $ || x^{*} - x_{k} | $ &  $ || x^{*} - x_{k} || / || x^{*} || $\\ \hline
10 & 1 & 97 & 0,000000000000 & 0,000000000000 & 475 & 0,000000000000 & 0,000000000000 \\ \hline 
10 & 2 & 98 & 0,000000000019 & 0,000000000001 & 475 & 0,000000000019 & 0,000000000001 \\ \hline 
10 & 3 & 100 & 0,000000000199 & 0,000000000010 & 475 & 0,000000000199 & 0,000000000010 \\ \hline 
10 & 4 & 99 & 0,000000000644 & 0,000000000033 & 475 & 0,000000000644 & 0,000000000033 \\ \hline 
10 & 5 & 98 & 0,000000008368 & 0,000000000426 & 475 & 0,000000008368 & 0,000000000426 \\ \hline 
10 & 6 & 100 & 0,000000094073 & 0,000000004794 & 475 & 0,000000094073 & 0,000000004794 \\ \hline 
100 & 1 & 9979 & 0,000000002233 & 0,000000000004 & 348250 & 0,000000002233 & 0,000000000004 \\ \hline 
100 & 2 & 9975 & 0,000000072988 & 0,000000000125 & 348250 & 0,000000072988 & 0,000000000125 \\ \hline 
100 & 3 & 9973 & 0,000000687986 & 0,000000001183 & 348250 & 0,000000687986 & 0,000000001183 \\ \hline 
100 & 4 & 9973 & 0,000003720160 & 0,000000006396 & 348250 & 0,000003720160 & 0,000000006396 \\ \hline 
100 & 5 & 9978 & 0,000002302613 & 0,000000003959 & 348250 & 0,000002302613 & 0,000000003959 \\ \hline 
100 & 6 & 9968 & 0,000107291983 & 0,000000184452 & 348250 & 0,000107291983 & 0,000000184452 \\ \hline 
1000 & 1 & 999750 & 0,000013726869 & 0,000000000751 & 334832500 & 0,000013726869 & 0,000000000751 \\ \hline 
1000 & 2 & 999747 & 0,000187991706 & 0,000000010289 & 334832500 & 0,000187991703 & 0,000000010289 \\ \hline 
1000 & 3 & 999752 & 0,000516185718 & 0,000000028251 & 334832500 & 0,000516185718 & 0,000000028251 \\ \hline 
1000 & 4 & 999762 & 0,016536243228 & 0,000000905049 & 334832500 & 0,016536243230 & 0,000000905049 \\ \hline 
1000 & 5 & 999733 & 0,047676899674 & 0,000002609414 & 334832500 & 0,047676899677 & 0,000002609414 \\ \hline 
1000 & 6 & 999774 & 0,306816218125 & 0,000016792423 & 334832500 & 0,306816218127 & 0,000016792423 \\ \hline 
\end{tabular}

Разлочие в количестве итераций обусловленно не учитыванием разложением матрицы на LU.

\begin{center}
	\begin{Large}
		\textbf{4. Выводы}
	\end{Large}
\end{center}

Исходя из проведенных опытов, можно сказать об одинаковых погрешностях метода Гаусса с выбором ведущего элемента и на основе LU-разложения, они оба имеют одинаковую асимптотическую сложность. При этом так как основную часть времени метод на основе LU, занимает само разложение матрицы, а значит при повторном исселдовании матрицы время работы метода будет меньше.

\begin{center}
	\begin{Large}
		\textbf{5. Приложения}
	\end{Large}
\end{center}

Реализация всех методов и всех вспомогательных классов выводящих резальтаты представленна в репозитории (\href{https://github.com/Sdmitrioul/metOpt/tree/main/skroba}{sdmitrioul})