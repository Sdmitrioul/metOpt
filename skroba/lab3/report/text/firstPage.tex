\begin{center}
	\begin{Large}
		\textbf{1. Постановка задачи}
	\end{Large}
\end{center}

\begin{itemize}
	\item Реализовать алгоритмы решения СЛАУ:
			\begin{enumerate}
				\item Метод решния СЛАУ на основе LU-разложения(форма хранения матрицы - профильная)
				\item Метод решения СЛАУ методом Гаусса с выбором ведущего элемента(форма матрицы - плотная, квадратичная)
			\end{enumerate}
	\item Написать генератор матриц:
		\begin{enumerate}
				\item С заданным числом обусловленности и размерностью.
				\item Матриц Гильберта
		\end{enumerate}
	\item Протестировать написанные методы на случайно сгенерированных матрицах.
	\item Сравнить два метода относительно точности и количество выполненных итераций.
\end{itemize}

\begin{center}
	\begin{Large}
		\textbf{2. Анализ методов}
	\end{Large}
\end{center}

\begin{itemize}
	\item \textbf{Метод 1:}\\
		$Ax = b \Leftrightarrow A = LU: LUx = b$ \\
		Суть метода, заключается в том, что исодная матрица $A$ представляется в виде произведения ниженетреугольной и верхнетреугольной матрицы: $L$ и  $U$ соответственно.  \\
		Далее решаются две подзадачи: \\
		\begin{equation*}
 		\begin{cases}
   			Ly = b \\
   			Ux = y
 		\end{cases}
	\end{equation*} \\
	Так как обе матрицы $L$ и $U$ являются треугольным, асимптотика решения этих уранений: $O(n^2)$, где n размерность пространства. \\
	Перед работой данного метода требуется привести матрицу к LU-виду, что в свою очередь занимает $O(n^3)$ времени.\\
	\item \textbf{Метод 2:}
	Суть метода заключается двух действиях: прямом и обратном ходе. Во время прямого хода последовательно находится ведущий элемент данного столбца - максимальный элемент, после чего все остальные не обработаннные строчки зануляют свой элемент этого столбца, тем самым по итогу получается треугольная матрица. После обратного хода вычисляется значения неизвестного вектора. Асимптотика алгоритма: $O(n^3)$- за счет прямого хода.
\end{itemize}

\bigskip

\begin{center}
	\begin{Large}
		\textbf{3. Результаты исследований}
	\end{Large}
\end{center}

\textbf{Параметры исследований:}
\begin{itemize}
	\item[-] Неизвестный вектор: $x^* = (0, 1, ..., n)$;
	\item[-]  $\epsilon = 10^{-20}$;
\end{itemize}
