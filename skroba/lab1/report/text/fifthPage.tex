\begin{center}
	\begin{Large}
		\textbf{4. Тестирование на многомодальных функциях}
	\end{Large}
\end{center}
Тестирование методов на функции: $f(x) = (x - 1) \cdot (x - 3) \cdot (x - 5) \cdot (x - 7) \cdot (x - 9) \rightarrow min : x \in [1; 10]$\\
\begin{center}
\begin{tabular}{|c|c|}
\hline
Dichotomy Method & 3,91213 \\ \hline
Golden Ratio Method & 3,91220 \\ \hline
Fibonacci Method & 3,91218 \\ \hline
Parabola Method & 3,91218 \\ \hline
Brant Combine Method & 8,28907 \\ \hline
Аналитическое решение & 8.2889 \\ \hline
\end{tabular}
\end{center}
\begin{figure}[h]
		\center{\includegraphics[width=0.7\linewidth]{static/multiModalFun.png}}
\end{figure}

\textbf{Вывод: } исходя из полученных резульатов делаем вывод, что данные методы не могут гарантировать нохождения глобального экстремума на отрезке, в случае если заданная им функция не является унимодальной.

\begin{center}
	\begin{Large}
		\textbf{5. Сравнение методов и вывод}
	\end{Large}
\end{center}

\begin{itemize}
\item Ожидается, что методу Брента требуется наименьшее количество операций для нахождения минимума, но вследствии того что функция данного варианта ведет себя при рассмотрении на заданном участе похоже на квадратическую метод парабол оказался более эффективным.
\item Метод брента окказался вторым по эффективности, требующий значительно меньше операций чем оставшиеся методы.
\item Метод Золотого сечения и метод Фиббоначи покказали себя более менее одинакого, но лучше чем метод дихотомии на разных унимодальных функкциях.
\item Все эти методы можно использоваться только для поиска мимимума для унимодальных функций.
\end{itemize}
P.S. На графики зависсимости количества вычислений метод дихотомии имеет малое кколичество вычислений из-за маленькой $\delta$, так как при больших, если $\epsilon < 1.0E-5$, количество операций сильно возрастает. 